\documentclass[12pt,a4paper]{report}

%\documentclass{book}
%\documentclass{article}

\usepackage{xeCJK}
\usepackage{makeidx}
\usepackage{multirow}
\usepackage{color}
\usepackage{graphicx}

\usepackage{uml}
%文泉驿 字体
%\setCJKmainfont{WenQuanYi Micro Hei}
%楷体
\setCJKmainfont{AR PL UKai CN}

\makeindex
\printindex
\setcounter{secnumdepth}{5}

\title{需求 \& 架构 \& 设计}
\author{PengPeng}

\begin{document}
\maketitle
\tableofcontents
\newpage

\chapter{概述}
为了指导Qe的设计,本文档总结Qe的需求、提出其架构、作出前期设计。随着项目的深入,该文档的设计部分应该越来越细化。
\newpage

\chapter{版本历史}
\begin{table}[!hbp]
\begin{center}
	\begin{tabular}{|l|l|l|l|}
		\hline
		版本 & 日期 & 作者 & 备注 \\
		\hline
		V1.0 & \today & 彭鹏 & 初始化版本 \\
		\hline
	\end{tabular}
	\caption{修订记录}
\end{center}
\end{table} 
\newpage

\chapter{参考文献}
\begin{enumerate}
	\item[*] 代码大全(中文第二版) 3.4节
\end{enumerate}
\newpage

\chapter{需求定义}
\begin{itemize}
	\item 问题定义: 目前业界没有较小巧易用的OpenGL ES2.0  3D引擎。
	\item 目的: 设计并实现可以商用的OpenGL ES2.0图形引擎。
	\item 输入: 以文件形式建模图形,文件格式细节定义于《文件\_规格V1.0.pdf》。
	\item 输出:
		\begin{itemize}
			\item 交互式3D图形输出 可执行文件 (类游戏实时渲染);
			\item 3D图形输出 avi文件 (类电影渲染)。
		\end{itemize}
	\item 要求:
		\begin{itemize}
			\item 嵌入式环境(实时渲染)EGL1.4、OpenGL ES2.0、Linux 标准键盘驱动(实时渲染);
			\item PC环境(实时渲染)Mesa;
			\item 尽可能高的可扩展性。
		\end{itemize}
	\item 限制:人员少、设计经验不丰富。
\end{itemize}
 
\chapter{架构设计}
本引擎架构见图\ref{子系统划分}:

\begin{figure}[!hbp]
\begin{center}
%\includegraphics[height=150mm,width=150mm]{subSystem.pdf}
\includegraphics{subSystem.pdf}
\end{center}
\caption{子系统划分\label{子系统划分}}
\end{figure}

如图\ref{子系统划分}本系统划分为Qe核心、源解析、控制、输出控制几部分:
\begin{description}
	\item[- Qe核心] 该系统是整个系统的核心,实现3D图形的渲染以及各种特效。
	\item[- 源解析] 该系统实现3D模型和场景的源文件解析,系统最终将支持所有主流3D文件,为了避免项目前期工作量过大而夭折,故本版极力减少工作量,仅实现Qe引擎专有文件的解析,3D模型和场景源文件细节见《文件\_规格V1.0.pdf》
	\item[- 控制] 该系统实现对渲染的场景的控制。目前对引擎的控制暂定为两类: 
		\begin{itemize}
			\item 实时控制使用键盘鼠标等作为控制方式实现类似游戏的实时渲染效果;
			\item 非实时控制使用控制脚本作为控制方式实现类似3D电影(图片)的渲染效果。
                \end{itemize}
	\item[- 输出] 出于减少工作量的考虑,该系统仅实现EGL方式的实时渲染。
\end{description}

\section{子系统交互}
子系统间的交互见图\ref{系统交互}:

\begin{figure}[!hbp]
\begin{center}
%\includegraphics[height=160mm,width=130mm]{systemComunicate.pdf}
\includegraphics{systemComunicate.pdf}
\end{center}
\caption{子系统交互\label{系统交互}}
\end{figure}

系统开始渲染后,3d模型以及场景数据从源文件中读取,源解析子系统将其解析为引擎内部3d对象传递给引擎核心。与此同时,控制解析子系统读取\footnote{若使用输入设备,例如鼠标做控制,则无需读取控制文件}控制文件\footnote{本引擎非核心子系统进行引擎外部数据与内部数据的转换,为了提高引擎的灵活性,并未将控制脚本整合入源文件。},并解析为引擎内部对象传递给引擎核心。引擎核心根据读取的控制对象和3d对象将其解析\footnote{可能包含3d矩阵计算,但出于效率的考虑尽可能利用OpenGL ES2.0提供的功能完成解析处理}为OpenGL ES2.0需要的数据格式传递给输出控制子系统后输出。
\section{源文件解析设计}
asymotote类图 调研
施工中\ldots
\section{核心设计}
施工中\ldots
\section{控制设计}
施工中\ldots
\section{输出设计}
施工中\ldots
\section{小结}
施工中\ldots
\newpage

\chapter{总结}
无
\newpage

\chapter{格式说明}

\begin{abstract}
这是摘要,留待扩展。
\end{abstract}

本文以下部分使用\LaTeXe{}最基本的配置完成文档的编写。 其中将介绍段落的划分、公式的使用、图形的插入\ldots 这篇文档中的内容将足以应付大多数应用。
\section{概述}
在编写文档时可以参考这部分的内容
\newpage

\section{表格}
\begin{table}[!hbp]
	\begin{tabular}{|c|c|c|c|c|}
		\hline
		\hline
		lable 1-1 & label 1-2 & label 1-3 & label 1 -4 & label 1-5 \\
		\hline
		label 2-1 & label 2-2 & label 3-3 & label 4-4 & label 5-5 \\
		\hline
		\multirow{2}{*}{Multi-Row} & \multicolumn{2}{|c|}{Multi-Column} & \multicolumn{2}{|c|}{\multirow{2}{*}{Multi-Row and Col}} \\
		\cline{2-3}
		& column-1 & column-2 & \multicolumn{2}{|c|}{}\\
		\hline
	\end{tabular}
	\caption{My first table}
\end{table} 
\newpage

\section{段落介绍}
\textcolor[rgb]{0.00, 0.50, 0.00}{带颜色的内容}
本章介绍段落的划分。\\
文档结束了。\\
列表项测试。
\begin{itemize}
	\item 第一大項,這裡是第一大項。
	\item 第二大項,這裡是第二大項。
		\begin{itemize}
			\item 第一小項,這裡是第一小項。
			\item 第二小項,這裡是第二小項。
		\end{itemize}
	\item 第三大項,這裡是第三大項。
	\item 第四大項,這裡是第四大項。
\end{itemize}
编号列表项测试。
\begin{enumerate}
	\item 第一大項,這裡是第一大項。
	\item 第二大項,這裡是第二大項。
		\begin{enumerate}
			\item 第一小項,這裡是第一小項。
			\item 第二小項,這裡是第二小項。
		\end{enumerate}
	\item 第三大項,這裡是第三大項。
	\item 第四大項,這裡是第四大項。
\end{enumerate}
描述列表项测试。
\begin{description}
	\item[第一大項] 這裡是第一大項。
	\item[第二大項] 這裡是第二大項。
		\begin{description}
			\item[第一小項] 這裡是第一小項。
			\item[第二小項] 這裡是第二小項。
		\end{description}
	\item[第三大項] 這裡是第三大項。
	\item[第四大項] 這裡是第四大項。
\end{description}
\newpage

\section{公式}

\begin{equation}
\epsilon > 0
\end{equation}

\begin{equation}
\lim_{n \to \infty}
\sum_{k=1}^n \frac{1}{k^2}
= \frac{\pi^2}{6}
\end{equation}

\newpage

\section{数学图像}
\setlength{\unitlength}{0.75mm}
\begin{picture}(60,40)
  \put(30,20){\vector(1,0){30}}
  \put(30,20){\vector(4,1){20}}
  \put(30,20){\vector(3,1){25}}
  \put(30,20){\vector(2,1){30}}
  \put(30,20){\vector(1,2){10}}
  \thicklines
  \put(30,20){\vector(-4,1){30}}
  \put(30,20){\vector(-1,4){5}}
  \thinlines
  \put(30,20){\vector(-1,-1){5}}
  \put(30,20){\vector(-1,-4){5}}
\end{picture}

\setlength{\unitlength}{1mm}
\newcommand{\wrt}[1]{\makebox(0,0)[c]{#1}}
\newcommand{\lline}[1]{\line(-1,0){#1}}
\newcommand{\rline}[1]{\line(1,0){#1}}
\newcommand{\uline}[1]{\line(0,1){#1}}
\newcommand{\dline}[1]{\line(0,-1){#1}}
\newcommand{\lvec}[1]{\vector(-1,0){#1}}
\newcommand{\rvec}[1]{\vector(1,0){#1}}
\newcommand{\uvec}[1]{\vector(0,1){#1}}
\newcommand{\dvec}[1]{\vector(0,-1){#1}}
\newsavebox{\condition}
\newsavebox{\process}
\newsavebox{\inputoutput}
\savebox{\process}(0,0){\thicklines
  \put(-10,-3){\framebox(20,6){}}
}
\savebox{\condition}(0,0){\thicklines
  \put(-10,0){\line(2,1){10}}
  \put(-10,0){\line(2,-1){10}}
  \put(10,0){\line(-2,1){10}}
  \put(10,0){\line(-2,-1){10}}
  }
\savebox{\inputoutput}(0,0){\thicklines
  \put(-10.5,-3){\rline{18}}
  \put(-10.5,-3){\line(1,2){3}}
  \put(10.5,3){\lline{18}}
  \put(10.5,3){\line(-1,-2){3}}
}
\begin{center}
\setlength{\unitlength}{0.75mm}
\begin{picture}(45,73)(20,-73)
	\thicklines
	\put(40,-19){\usebox{\process}}
	\put(40,3){\wrt{开始}}
	\put(40,0){\dvec{5}}
\put(40,-8){\usebox{\inputoutput}} \put(40,-8){\wrt{输入$m,\,n$}}
\put(40,-11){\dvec{5}} \put(40,-19){\usebox{\process}}
\put(40,-19){\wrt{$i=1$}} \put(40,-22){\dvec{8}}
\put(40,-33){\usebox{\process}} \put(40,-33){\wrt{$a=m\times i$}}
\put(40,-36){\dvec{5}} \put(40,-46){\usebox{\condition}}
\put(40,-46){\wrt{$n$整除$a$?}} \put(40,-51){\dvec{5}}
\put(41,-53){\makebox(0,0)[l]{是}}
\put(40,-59){\usebox{\inputoutput}} \put(40,-59){\wrt{输出$a,\,i$}}
\put(40,-62){\dvec{5}}
\put(40,-70){\oval(15,6)}\put(40,-70){\wrt{结束}}
\put(30,-46){\line(-1,0){15}} \put(28,-45){\makebox(0,0)[b]{否}}
\put(15,-46){\uvec{10}} \put(15,-33){\usebox{\process}}
\put(15,-33){\wrt{$i=i+1$}} \put(15,-30){\line(0,1){5}}
\put(15,-25){\rvec{25}}
\end{picture}
\end{center}

\begin{figure}[!hbp]
\setlength{\unitlength}{1mm}
\begin{picture}(136, 100)
	%绘制坐标线
	\linethickness{0.01mm}
	\put(0, 0){\framebox(136, 100){}}
	\linethickness{0.01mm}
	\multiput(0,0)(13.6,0){10}{\line(0,1){100}}
	\multiput(0,0)(0, 10){10}{\line(1, 0){136}}
	\linethickness{0.1mm}
	\put(10, 94){\framebox(20, 6){源文件}}
	\put(10, 84){\framebox(20, 6){源解析}}
	\put(116, 68){\framebox(20, 6){控制文件}}
	\put(90, 62){\framebox(20, 18){控制解析}}
	\put(35, 0){\framebox(20, 6){视频文件}}
	\put(30, 10){\framebox(60, 6){输出控制}}
	\put(65, 0){\framebox(20, 6){显示器}}
	\put(30, 40){\framebox(40, 30){\LARGE引擎核心}}
	\put(10, 10){\framebox(100, 80){}}
	\put(40, 65){\makebox(40,30){\huge Qe引擎}}

	\thinlines
	\put(20, 94){\vector(0, -1){4}}
	\put(20, 84){\line(0, -1){30}}
	\put(20, 54){\vector(1, 0){10}}
	\put(116, 71){\vector(-1, 0){6}}
	\put(80, 54){\line(0, 1){17}}
	\put(80, 71){\line(1, 0){10}}
	\put(80, 54){\vector(-1, 0){10}}
	\put(60, 30){\vector(0, -1){14}}
	\put(60, 30){\line(-1, 0){10}}
	\put(50, 30){\line(0, 1){10}}
\end{picture}
\caption{子系统交互\label{子系统交互}}
\end{figure}
\newpage

\section{整段English试试}
Hello, I'm brother peng!
\newpage

\section{脚注}
测试脚注\footnote{我是脚注.}


\section{边注}
测试边注\marginpar{我是边注}

\end{document}

